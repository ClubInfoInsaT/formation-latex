\documentclass[10pt]{article}
\usepackage[utf8]{inputenc}
\usepackage[T1]{fontenc}
\usepackage[french]{babel}
\fontshape{n}
\pagestyle{empty}
\usepackage[left=4cm,right=4cm,top=2cm,bottom=2cm]{geometry}
\usepackage{amsmath, amsfonts, fancyhdr, graphicx, float, hyperref, xcolor, framed, dsfont, latexsym , upgreek , datetime , xcolor , caption, listings , hyperref }
\usepackage[newfloat]{minted}

\pagestyle{fancy}
\fancyhf{}
\setlength{\footskip}{32.05098pt}

\newenvironment{cadre}[1]{%
    \definecolor{bord}{HTML}{#1}
    \colorlet{fond}{bord!20}
    \def\FrameCommand{{\color{bord}\vrule width 3pt}\colorbox{fond}}
    \MakeFramed {\advance\hsize-\width \FrameRestore}
    \noindent
}{\endMakeFramed}

\usepackage[most]{tcolorbox}
\usepackage{fontawesome5} % Pour l'icône d'attention

% Définir l'environnement warning
\newtcolorbox{warningbox}{
  colback=white, % Couleur de fond
  colframe=red!80, % Couleur de la bordure
  fonttitle=\bfseries, % Style du titre
  title=\faExclamationTriangle\hspace{0.5em}Attention !, % Titre de la boîte
  enhanced, % Permet l'utilisation de tcolorbox avec \newtcolorbox
  attach boxed title to top left={yshift=-2mm, xshift=3mm}, % Position du titre
  boxed title style={colback=red, sharp corners}, % Style du titre de la boîte
}

\newcommand{\R}{\mathbb{R}}
\newcommand{\C}{\mathbb{C}}
\newcommand{\N}{\mathbb{N}}
\newcommand{\K}{\mathbb{K}}


\title{titre du document}
\author{auteur}
\date{\today}


\rhead{entete 2}
\lhead{entete 1}

% R : right, L : Left, C: center || \thepage pour le numéro de page, sinon un texte ou une image
\fancyfoot[C]{\thepage} 


\begin{document}
    \maketitle
    \tableofcontents % table des matières
    \newpage % saut de page
    
    \section{Une première section}
    
        \textbf{ceci est un texte en gras}
    
        \begin{warningbox}
            attention ceci est pas bien
        \end{warningbox}

        \begin{cadre}{FFC63D}
            ceci est un théorème en jaune $\forall (x,y,z,w) \in \R \times \C \times \K \times \N$
        \end{cadre}

        \begin{cadre}{00FF00}
            ceci est un autre théorème en vert $\displaystyle \int_\R x^2 dx$
        \end{cadre}
    
    \section{Une seconde section}
    
    \subsection{Une sous section}
    
    Une énumération :
    \begin{enumerate}
        \item voila
        \item bof
    \end{enumerate}
    \vspace{2em}
    Une liste :
    \begin{itemize}
        \item voila
        \item bof
    \end{itemize}
    
    \subsection{Une seconde sous section}

    et on commence à écrire du texte \\
    \textbf{ceci est un texte en gras}\\
    \textit{Ceci est en italique}\\
    \underline{ceci est souligné}
\end{document}